\documentclass[review,3p,times,authoryear,12pt]{elsarticle}
%\usepackage{amsmath,amsthm,amssymb}
%\usepackage{algorithm}
%\usepackage{graphicx,subfigure}
%\usepackage{clrscode3e}
\usepackage{enumerate}
%\usepackage{multirow,endfloat}
%\graphicspath{{figure/}}
%\newtheorem{proposition}{Proposition}

\begin{document}
\begin{frontmatter}
\newpage
\title{A Heuristic for the Stowage Stack Minimization Problem with K-Rehandle Constraint under the circular route}
\begin{abstract}
% to do:
write the abstract
\end{abstract}
\begin{keyword}
Containership stowage planning\sep circular routes\sep stack minimization \sep K-rehandles \sep constructive heuristic
\end{keyword}
\end{frontmatter}

\section{Introduction}
% to do:
%1.introduce the importance of containership transpotation.
%2.introduce the container stowage planning.
%3.introduce the circular routes and ship structure.
%4.introduce what we did and what the contributions are.
%5.introduce the contexts of next chapters.

The increasing demand for maritime transportation of containers have led to the construction of numerous mega container vessels, many of which have the capacity to carry more than 18,000 twenty-foot equivalent units (TEU) containers.
A stack is obtained by stacking containers vertically.
Several stacks in a row form a bay, and bays are placed side by side to form a container block.
In a general case, container ship serves many different ports on each voyage.
A stowage planning for container ship made at one port must take account of the influence on subsequent ports.
Therefore, the complexity of stowage planning problem increases due to its multi-ports nature.

Under the annular route, there are different kinds of containers at each port to be loaded. 
Under the linear route in \cite{wang2014stowage}, the destination port of container is larger than the origin port of containers
Due to the characteristic of circular routine, we have to design more data structure to store different container types and simulate different loading and unloading situations.
\section{Literature review}
% to read and review literatures


\section{Problem description and properties}
In our stowage stack minimization problem with K-Rehandle constraint under the circular route, the whole route of the ship is investigated; in particular, different sets of containers must be loaded for being shipped to the next ports at each port of the route.
The sequence of two handling operations has an important influence on the effectiveness of a stowing plan: first, the import containers must be unloaded from the ship, then the export containers can be loaded.
Given a ship with its structural characteristics, its route, described by a circular sequence of ports to be visited, and its current cargo, the problem consists in defining the stowage plan for a given set of containers that differ for the loading and destination port, so that all the containers are loaded on board, while the structural and operative constraints are satisfied and the number of stacks used on the vessel at the ports for loading/unloading operations is minimized.
Each ship travels on a circular route with P ports and the transport demand is randomly generated in such a way that for each origin port in the route of a ship.
For example, if P equals to 6, when planning the stowage for port 5, the ship has on board a cargo deriving from loading operations executed at ports 1, 2, 3 and 4. At port 5, after that the unloading operations are executed, the loading process regards containers bound for port 6, 1��, 2��, 3�� and 4��, where 1��, 2��, 3�� and 4�� denote the port 1, 2, 3 and 4 reached during the second round of the ship.
For each containership, five instances have been generated; each instance differs from the others for the transport demand to satisfy.
Anyway, all instances have been generated in such a way to stress the capability of the heuristic approach to obtain feasible and effectiveness solutions in a short amount of time.
We divide our instances into three parts according to the size of limit height: Small Ship with H equals to 4 and N equals to 100; Medium Ship with H equals to 8 and N is selected from {400, 1000}; Large Ship with H equals to 12 and N equals to 2000.
There are some assumptions for the convenience of our research:
\begin{itemize}
\item	All the containers have the same size, twenty-foot equivalent units (TEU) containers;
\item	Containers at each port for each loop have the same quantities and types;
\item	The vessel has a limit height considering the security and balance of the vessel;
\item	After the unloading and loading operations at each port, there are at most K rehandles exist;
\item	The voyage should stop at a certain loop after we have found the convergence.
\item	Other constraints to ensure the security of vessel are satisfied.
\end{itemize}

\subsection{Integer model}


\section{Methodology}
We have constructed a heuristic algorithm which greedy rules are adapted.
There are main three procedures in our algorithm: unloading, sorting and loading.
Considering the existing of re-handles, our strategies to handle loading and unloading operations are different.
The rules become strict when it comes to meeting the conditions that there are certain re-handles given.
However, the main ideas are extremely similar.
The stacks on the vessel are divided into different categories according to whether it will occur re-handle when the loading container is loaded into the stack.
Hereafter, we give them different priority to reduce the number of used stacks as much as possible.
The priority of them is following:
\begin{itemize}
\item Partial stacks with no re-handles occurring.
\item Partial stacks with re-handles occurring.
\item Empty stacks.
\end{itemize}
Obviously, there won��t be the second one if the given re-handles run out.




\section{Experiments and Analysis}


\section{Conclusion}

\bibliographystyle{apalike2}
\bibliography{Cir_SSMP}
\end{document}
